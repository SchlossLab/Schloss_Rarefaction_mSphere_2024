% Options for packages loaded elsewhere
\PassOptionsToPackage{unicode}{hyperref}
\PassOptionsToPackage{hyphens}{url}
%
\documentclass[
]{article}
\usepackage{amsmath,amssymb}
\usepackage{lmodern}
\usepackage{iftex}
\ifPDFTeX
  \usepackage[T1]{fontenc}
  \usepackage[utf8]{inputenc}
  \usepackage{textcomp} % provide euro and other symbols
\else % if luatex or xetex
  \usepackage{unicode-math}
  \defaultfontfeatures{Scale=MatchLowercase}
  \defaultfontfeatures[\rmfamily]{Ligatures=TeX,Scale=1}
\fi
% Use upquote if available, for straight quotes in verbatim environments
\IfFileExists{upquote.sty}{\usepackage{upquote}}{}
\IfFileExists{microtype.sty}{% use microtype if available
  \usepackage[]{microtype}
  \UseMicrotypeSet[protrusion]{basicmath} % disable protrusion for tt fonts
}{}
\makeatletter
\@ifundefined{KOMAClassName}{% if non-KOMA class
  \IfFileExists{parskip.sty}{%
    \usepackage{parskip}
  }{% else
    \setlength{\parindent}{0pt}
    \setlength{\parskip}{6pt plus 2pt minus 1pt}}
}{% if KOMA class
  \KOMAoptions{parskip=half}}
\makeatother
\usepackage{xcolor}
\usepackage[margin=1.0in]{geometry}
\usepackage{longtable,booktabs,array}
\usepackage{calc} % for calculating minipage widths
% Correct order of tables after \paragraph or \subparagraph
\usepackage{etoolbox}
\makeatletter
\patchcmd\longtable{\par}{\if@noskipsec\mbox{}\fi\par}{}{}
\makeatother
% Allow footnotes in longtable head/foot
\IfFileExists{footnotehyper.sty}{\usepackage{footnotehyper}}{\usepackage{footnote}}
\makesavenoteenv{longtable}
\usepackage{graphicx}
\makeatletter
\def\maxwidth{\ifdim\Gin@nat@width>\linewidth\linewidth\else\Gin@nat@width\fi}
\def\maxheight{\ifdim\Gin@nat@height>\textheight\textheight\else\Gin@nat@height\fi}
\makeatother
% Scale images if necessary, so that they will not overflow the page
% margins by default, and it is still possible to overwrite the defaults
% using explicit options in \includegraphics[width, height, ...]{}
\setkeys{Gin}{width=\maxwidth,height=\maxheight,keepaspectratio}
% Set default figure placement to htbp
\makeatletter
\def\fps@figure{htbp}
\makeatother
\setlength{\emergencystretch}{3em} % prevent overfull lines
\providecommand{\tightlist}{%
  \setlength{\itemsep}{0pt}\setlength{\parskip}{0pt}}
\setcounter{secnumdepth}{-\maxdimen} % remove section numbering
\newlength{\cslhangindent}
\setlength{\cslhangindent}{1.5em}
\newlength{\csllabelwidth}
\setlength{\csllabelwidth}{3em}
\newlength{\cslentryspacingunit} % times entry-spacing
\setlength{\cslentryspacingunit}{\parskip}
\newenvironment{CSLReferences}[2] % #1 hanging-ident, #2 entry spacing
 {% don't indent paragraphs
  \setlength{\parindent}{0pt}
  % turn on hanging indent if param 1 is 1
  \ifodd #1
  \let\oldpar\par
  \def\par{\hangindent=\cslhangindent\oldpar}
  \fi
  % set entry spacing
  \setlength{\parskip}{#2\cslentryspacingunit}
 }%
 {}
\usepackage{calc}
\newcommand{\CSLBlock}[1]{#1\hfill\break}
\newcommand{\CSLLeftMargin}[1]{\parbox[t]{\csllabelwidth}{#1}}
\newcommand{\CSLRightInline}[1]{\parbox[t]{\linewidth - \csllabelwidth}{#1}\break}
\newcommand{\CSLIndent}[1]{\hspace{\cslhangindent}#1}
\usepackage{booktabs}
\usepackage{longtable}
\usepackage{array}
\usepackage{multirow}
\usepackage{wrapfig}
\usepackage{float}
\usepackage{colortbl}
\usepackage{pdflscape}
\usepackage{tabu}
\usepackage{threeparttable}
\usepackage{threeparttablex}
\usepackage[normalem]{ulem}
\usepackage{makecell}
\usepackage{setspace}
\doublespacing
\usepackage[left]{lineno}
\linenumbers
\modulolinenumbers
\usepackage{helvet} % Helvetica font
\renewcommand*\familydefault{\sfdefault} % Use the sans serif version of the font
\usepackage[T1]{fontenc}
\usepackage[shortcuts]{extdash}
\usepackage{booktabs}
\usepackage{longtable}
\usepackage{array}
\usepackage{multirow}
\usepackage{wrapfig}
\usepackage{float}
\usepackage{colortbl}
\usepackage{pdflscape}
\usepackage{tabu}
\usepackage{threeparttable}
\usepackage{threeparttablex}
\usepackage[normalem]{ulem}
\usepackage{makecell}
\usepackage{xcolor}
\ifLuaTeX
  \usepackage{selnolig}  % disable illegal ligatures
\fi
\IfFileExists{bookmark.sty}{\usepackage{bookmark}}{\usepackage{hyperref}}
\IfFileExists{xurl.sty}{\usepackage{xurl}}{} % add URL line breaks if available
\urlstyle{same} % disable monospaced font for URLs
\hypersetup{
  hidelinks,
  pdfcreator={LaTeX via pandoc}}

\author{}
\date{\vspace{-2.5em}}

\begin{document}

\hypertarget{rarefy-your-data}{%
\section{Rarefy your data}\label{rarefy-your-data}}

\vspace{20mm}

\textbf{Running title:} Rarefy your data

\vspace{20mm}

Patrick D. Schloss\({^\dagger}\)

\vspace{40mm}

\({\dagger}\) To whom corresponsdence should be addressed:

\href{mailto:pschloss@umich.edu}{pschloss@umich.edu}

Department of Microbiology \& Immunology

University of Michigan

Ann Arbor, MI 48109

\vspace{20mm}

\textbf{Research article}

\newpage

\hypertarget{abstract}{%
\subsection{Abstract}\label{abstract}}

\hypertarget{importance}{%
\subsection{Importance}\label{importance}}

\newpage

\hypertarget{introduction}{%
\subsection{Introduction}\label{introduction}}

\begin{itemize}
\tightlist
\item
  Motivation

  \begin{itemize}
  \tightlist
  \item
    Problem of uneven sampling effort
  \end{itemize}
\item
  What is rarefaction? History, reason for rarefaction

  \begin{itemize}
  \tightlist
  \item
    Repeated down sampling of datasets to a common number of
    observations to calculate the average value to ascertain the
    expected value of a metric for the metric under study; typically
    richness
  \item
    Control for unneven sampling effort
  \item
    Methods vary in their sensitivity to uneven sampling
  \item
    Compositional data analysis
  \end{itemize}
\item
  Reasons behind ``rarefaction is inadmissable''

  \begin{itemize}
  \tightlist
  \item
    Weird simulation
  \end{itemize}
\item
  Alternative approaches and claims

  \begin{itemize}
  \tightlist
  \item
    sampling invariance
  \end{itemize}
\item
  Goal of this study
\end{itemize}

This analysis included 16S rRNA gene sequence data from from 12 studies
that characterized the variation in bacterial communities from diverse
environments (Table 1). The original studies generated the sequence data
by pooling separate PCR products that were generated by amplifying the
V4 region of the 16S rRNA gene from the bacterial DNA in multiple
samples. Because pooling equimolar quantities of DNA is frought with
difficulties, it was common to observe wide variation in the number of
sequences in each sample (Table 1 and Figure S1).

\hypertarget{results}{%
\subsection{Results}\label{results}}

\textbf{\emph{Without rarefaction, metrics of alpha diversity are
sensitive to sampling effort.}} To test the sesitivity of various
approaches of measuring alpha diversity to sampling effort, I generated
null models for each study. Under a null model, each community from the
same study would be expected to have the same alpha diversity regardless
of the sampling effort. I measured the richness of the communities in
each study without any correction, using scaled ranked subsampling (SRS)
normalized OTU counts, with estimates based on non-parametric and
parametric approaches, and using rarefaction (e.g.~Figure S2). For each
study, all of the approaches, except for rarefaction, showed a strong
correlation between richness and the number of sequences in the sample
(Figure 1A). Next, I assessed diversity using the Shannon diversity
index and the inverse Simpson diversity index without any correction,
using normalized OTU counts, and rarefaction; I also used a
non-parametric estimator of Shannon diversity. The correlation between
sampling depth and the diversity metric was not as strong as it was for
richness and the inverse Simpson diversity values were less sensitive
than the Shannon diversity values; however, the correlation to the
rarefied diversity metrics were the lowest for all of the metrics and
studies (Figure 1A). The rarefied alpha-diversity metrics consistently
demonstrated a lack of sensitivity to sampling depth.

\textbf{\emph{Without rarefaction, metrics of beta diversity are
sensitive to sampling effort.}} To test the sesitivity of various
approaches of measuring beta diversity to sampling effort, I used the
same null models used for studying the sensitivity of alpha diversity.
Under a null model, the ecological distance between any pair of samples
would be the same regardless of the difference in the number of
sequences observed in each sample (e.g., Figure S3). First, I calculated
the Jaccard distance coefficient between all pairs of communities within
a study. The Jaccard distance coefficient is the fraction of OTUs that
are unique to either community and does not account for the abundance of
the OTUs. Jaccard distances were calcualted using the uncorrected OTU
counts, with rarefaction, relative abundances, and following
normalization using cumulative sum scaling (CSS) and SRS. Only the
rarefied distances showed a lack of sensitivity to sampling effort
(Figure 1B). Second, I analyzed the sensitivity of the Bray-Curtis
distance coefficient, which is a popular metric that incorporates the
abundance of each OTU. Similar to what I observed with the Jaccard
coefficient, only the rarefied data showed a lack of sensitivity to
sampling effort (Figure 1B). Third, I calcualted the Euclidean distance
on raw OTU counts where the central log-ratio (CLR) was calculated
(i.e., Aitchison distances) by ignoring OTUs in samples with zero counts
(Robust CLR), adding a pseudocount of one to all OTU counts prior to
calculating the CLR (One CLR), adding a pseudocount of one divided by
the total number of sequences obtained for the community (Nudge CLR),
and imputing the value of zero counts (Zero CLR). The Aitchison
distances were all strongly sensitive to sampling effort (Figure 1B).
Finally, I used the variance stabilization technique (VST) from DeSeq2
prior to calculating Euclidean distances. Again, there was a strong
sensitivity to sampling effort (Figure 1B). Although Euclidean distances
are not typically used on raw or rarefied count data in ecology,
rarefied Euclidean distances were not sensitive to sampling effort.
Across each of the beta diversity metrics and approaches used to account
for uneven sampling effort and sparsity, rarefaction was the least
sensitive approach to differences in sampling effort.

\textbf{\emph{Rarefaction limits the detection of false positives when
sampling effort and treatment group are confounded.}} Next, I
investigated the impact of the various strategies and metrics on falsely
detecting a significant difference using the the same communities
generated from the null model in the analysis of alpha and beta
diversity metrics. To test for differences in alpha and beta diversity I
used the non-parametric Wilcoxon test and non-parametric
permutation-based multivariate analysis of variance (PERMANOVA). First,
within each study, I randomly assigned each sample to one of two
treatment groups. My expectation was that approximately 5 of the 100
(5\%) random tests for each comparison would yield a significant test
result. Indeed, for each study and alpha and beta diversity metric and
strategy for accounting for uneven sampling, approximately 5\% of the
tests yielded a significant result (Figure 2). Second, within each
study, I assigned samples with more than the median number of sequences
per sample to one treatment group and the rest to another treatment
group. If there was no sensitivity to sampling effort, I would have
expected that 5\% of the tests would yield a significant result. In
fact, only the rarefied data consistently resulted in a 5\% false
positive rate for alpha and beta diversity metrics (Figure 2). These
results align with the observed sensitivity of alpha and beta diversity
metrics to sampling effort and underscore the value of rarefaction.

\textbf{\emph{Rarefaction preserves the statistical power to detect
differences between treatment groups.}} To assess the impact of
different approaches to control for uneven sampling effort I performed
two additional simulations. In the first simulation, for each study
communities were randomly assigned to one of two treatment groups.
Communities in the first treatment group were generated by sampling from
the null distribution. Samples in the second treatment group were
generated by perturbing the null distribution by increasing the relative
abundance of 10\% of the OTUs by 5\%. These values were determined after
empirically searching for conditions that resulted in a large fraction
of the randomizations yieleding a significant result across most of the
studies. The fraction of tests that yielded a significant test was a
measure of the statistical power for the test. The power to detect
differences in richness by all approaches was low (Figure 4A). This was
likely because the approach for generating the perturbed community did
not necessarily change the number of OTUs in each treatment group.
Regardless, the simulations testing differencse in richness using the
Rice and Stream datasets had the greatest power when the richness data
were rarefied. To explore this further, in a second simulation the
second treatment group was perturbed by removing 3\% of the OTUs from
the model. As suggested by the first simulation, the rarefied richness
data had a higher statistical power than the other approaches when
measuring richness (Figure 5). The simulations testing the power to
detect differences in Shannon diversity also showed that rarefied data
performed other methods (Figure 4A). When testing for differences in the
Inverse Simpson diversity index the the difference between rarefaction
and the other methods was negligible (Figure 4A). For tests of beta
diversity I found that rarefaction was the most reliable approach to
maintain statistical power to detect differences between two
communities. Among the tests using the Jaccard and Bray-Curtis metrics,
raw count data and CSS normalized data had little power relative to
rarefied, relative abundance, and SRS normalized data. The differences
in power between rarefied, relative abundance, and SRS normalized data
was small, but if there were differences, the power obtained using
rarefied data was greater than the other methods. Among the tests using
Euclidean distances, using raw counts and CLR and DeSeq2 transformed
data had little power relative to the distances calcualted using
rarefied and relative abundance data. This power-based analysis of the
simulated communities using different methods of handling uneven sample
sizes demonstrated the value of rarefaction for preserving the
statistical power to detect differences between treatment groups for
measures of alpha and beta diversity.

\textbf{\emph{Increased rarefaction depth reduces intra-sample variation
in alpha and beta diversity.}} Once concern with rarefying communities
is the perceived loss of sequencing information when more a large
fraction of data appears to be removed when the community with the
greatest sequencing depth is rarefied to the size of the community with
the least (e.g., 99\% with the Bioethanol dataset). To assess the
sensitivity of alpha and beta diversity metrics to rarefaction depth, I
again used the dataset generated using the null models, but rarefied
each community to varying sampling depths (Figure 6). The richness
values increased with sampling effort as rare OTUs would continue be
detected. In contrast, the Shannon diversity and Bray-Curtis values
plateaued with increased sampling effort. This result was expected since
increased sampling would lead to increased precision in the measured
abundance of OTUs. Next, I measured the coefficient of variation (i.e.,
the mean divided by the standard deviation) between samples for
richness, Shannon diversity, and Bray-Curtis distances. Although the
richness values appeared to increased unbounded with smapling effort,
the coefficient of variation for each dataset was relatively stable. In
general, the coefficient of variation increased slightly with sampling
depth only to decline once smaller samples were removed from the
analysis at higher sampling depths. Interestingly, the coefficient of
variation between Shannon diveristy values decreased towards zero with
increased sampling effort and the coefficient of variation between
Bray-Curtis distances tended to increased. Regardless, the coefficients
of variation were relatively small.

\textbf{\emph{Most studies have a high level of sequencing coverage.}}
To explore the concern over loss of sequencing depth further, I
calculated the Good's coverage for the observed data. The median
coverage for each dataset ranged between 89.4 and 99.8\% for the
Seagrass and Human datasets, respectively (Figure 7). When I rarefied
each dataset to the size of the smallest community in the dataset, with
the exception of the Seagrass, Rice, and Stream datasets, the median
coverage for the rarefied communities was still greater than 90\%. These
results suggest that most studies had a level of sequencing coverage
that aligned with the diversity of the communities. Next, I used the
null model for each dataset to ask how much sequencing effort was
required to obtain higher levels of coverage. To obtain 95 and 99\%
coverage, an average of 2.70 and 101.20-fold more sequence data was
estimated to be required than was required to obtain 90\% coverage,
respectively (Figure 7). As suggested by the simulated coverages curve
in Figure 7, these estimates are conservative. Regardless, the
sequencing effort required to acheive higher sequencing depth would
likely limit the number of samples that could be sequenced when
controlling for costs. Although it may be disconcerting to rarefy to a
sequencing depth that is considerably lower than that obtained for the
best sequenced community in a study, sequencing coverage for many
environments is probably adequate even at the lower sequencing depth. Of
course, the results above have demonstrated that rarefaction is
necessary to avoid problems with making inferences.

\hypertarget{discussion}{%
\subsection{Discussion}\label{discussion}}

\begin{itemize}
\tightlist
\item
  Rarefy your data
\item
  Problems with recommended methods\ldots{}

  \begin{itemize}
  \tightlist
  \item
    Many recommended methods are borrowed from gened expression analysis
  \item
    Meaning of zeroes in data - structural vs.~below limit of detection
  \end{itemize}
\item
  Factors that determine what number of sequences to rarefy to
\item
  Need better methods of pooling libraries that result in more even
  distribution of sequences across samples
\item
  Rarefy your data
\end{itemize}

\hypertarget{materials-and-methods}{%
\subsection{Materials and Methods}\label{materials-and-methods}}

\textbf{Choice of datasets.} The specific studies were selected because
their data was publicly accessible through the Sequence Read Archive,
the original investigators sequenced the V4 region of the 16S rRNA gene
using paired 250 nt reads, and my previous familiarity with the data.
The use of paired 250 nt reads to sequence the V4 region resulted in a
near complete two-fold overap of the V4 region resulting in high quality
contigs with a low sequencing error rate (1). These data were processed
through the standard sequence curation pipeline to generate operational
taxonomic units (OTUs) using the mothur software package (1, 2).

\textbf{Null community model.} Null models were generated by randomly
assigning each sequence in the study to an OTU and sample while keeping
constant the number of sequences per sample and the total number of
sequences in each OTU. Because the construction of the null models was a
stochastic process, 100 replicates were geneated for each study.

\textbf{10\%/5\% model}

\textbf{Remove model}

\textbf{Null treatment model}

\textbf{Size-based treatment model}

\textbf{Data availability.}

\textbf{Reproducible data analysis.}

\textbf{\emph{doi: 10.1038/nmeth.2658.}} Should CSS be used in alpha
diversity analysis? No.~The richness and shannon diversity values were
no different from those obtained with the raw abundances. From the
paper, ``The relative proportion of the features is unaffected by the
normalization''. CSS paper refers to relative abundance approach as
total-sum normalization (TSN)

Robust CLR - remove zero counts and calculate CLR One CLR - add a
pseudocount of 1 to all observations Nudge CLR - add a pseudocount of
1/total number of sequences in sample Zero CLR - Impute the value of
zeroes using zCompositions package

\textbf{\emph{\url{https://edepot.wur.nl/547087}.}} ``Log- ratio PCA is
designed to give results that are library size- independent. However, as
we demonstrated mathematically and with examples based on simulated and
real data, log- ratio PCA becomes library size- dependent, if there are
many infrequent taxa (many zeroes) and library sizes differ largely. In
this situation, the row centring used in log- ratio PCA brings an effect
of r (the row mean of the log- transformed counts) in the clr-
transformed matrix. Note that this effect is irrespective of whether or
not these infrequent taxa are genuine or due to sequencing noise or
allocation error. This library size dependence is unexpected in the
sense that, after applying the clr, the transformed matrix is free of
the effect of the row totals for strictly positive data
(yij\textgreater0 for all i and j). We additionally demonstrate that
library size variability causes a loss in power in detecting an effect
of x with log- ratio RDA. If there is additionally a correlation between
treatment and the library size, the type 1 error for detecting the
effect of x can be seriously inflated.''

\textbf{\emph{\url{https://www.nature.com/articles/s41522-020-00160-w}.}}
``An important characteristic of a feature table is that it is typically
sparse, sometimes as many as \textasciitilde90\% are zero entries21,
which creates a challenge for analyzing rare taxa. A quick and simple
strategy to deal with excess zeros is to add a small positive constant
(e.g.~1) called pseudo-count14,22 to each cell of the feature table. The
addition of a pseudo-count becomes necessary when using methods of
analysis that require log transformation of the observed counts. Even
though adding a pseudo-count is simple and widely used, the choice of
the pseudo-count is ad hoc. Studies have shown that differential
abundance or clustering results could be sensitive to the choice of
pseudo count23,24. Although different values of pseudo counts have been
discussed in the literature23,24,25,26, to the best of our knowledge,
there is no consensus on how to choose the optimal value. Other
strategies involve modeling zero counts by some probability
models21,27.''

\url{https://academic.oup.com/bioinformatics/article/34/16/2870/4956011}
\url{https://www.ncbi.nlm.nih.gov/pmc/articles/PMC6755255/}
\vspace{10mm}

\textbf{Acknowledgements.}

\newpage

\hypertarget{references}{%
\subsection{References}\label{references}}

\setlength{\parindent}{-0.25in}
\setlength{\leftskip}{0.25in}

\noindent

\hypertarget{refs}{}
\begin{CSLReferences}{0}{1}
\leavevmode\vadjust pre{\hypertarget{ref-Kozich2013}{}}%
\CSLLeftMargin{1. }%
\CSLRightInline{\textbf{Kozich JJ}, \textbf{Westcott SL}, \textbf{Baxter
NT}, \textbf{Highlander SK}, \textbf{Schloss PD}. 2013. {Development of
a dual-index sequencing strategy and curation pipeline for analyzing
amplicon sequence data on the MiSeq Illumina sequencing platform}.
Applied and environmental microbiology \textbf{79}:5112--5120.}

\leavevmode\vadjust pre{\hypertarget{ref-Schloss2009}{}}%
\CSLLeftMargin{2. }%
\CSLRightInline{\textbf{Schloss PD}, \textbf{Westcott SL},
\textbf{Ryabin T}, \textbf{Hall JR}, \textbf{Hartmann M},
\textbf{Hollister EB}, \textbf{Lesniewski RA}, \textbf{Oakley BB},
\textbf{Parks DH}, \textbf{Robinson CJ}, \textbf{Sahl JW}, \textbf{Stres
B}, \textbf{Thallinger GG}, \textbf{Horn DJV}, \textbf{Weber CF}. 2009.
Introducing mothur: Open-source, platform-independent,
community-supported software for describing and comparing microbial
communities. Applied and Environmental Microbiology
\textbf{75}:7537--7541.
doi:\href{https://doi.org/10.1128/aem.01541-09}{10.1128/aem.01541-09}.}

\leavevmode\vadjust pre{\hypertarget{ref-Li2015}{}}%
\CSLLeftMargin{3. }%
\CSLRightInline{\textbf{Li Q}, \textbf{Heist EP}, \textbf{Moe LA}. 2015.
Bacterial community structure and dynamics during corn-based bioethanol
fermentation. Microbial Ecology \textbf{71}:409--421.
doi:\href{https://doi.org/10.1007/s00248-015-0673-9}{10.1007/s00248-015-0673-9}.}

\leavevmode\vadjust pre{\hypertarget{ref-Baxter2016}{}}%
\CSLLeftMargin{4. }%
\CSLRightInline{\textbf{Baxter NT}, \textbf{Ruffin MT}, \textbf{Rogers
MAM}, \textbf{Schloss PD}. 2016. Microbiota-based model improves the
sensitivity of fecal immunochemical test for detecting colonic lesions.
Genome Medicine \textbf{8}.
doi:\href{https://doi.org/10.1186/s13073-016-0290-3}{10.1186/s13073-016-0290-3}.}

\leavevmode\vadjust pre{\hypertarget{ref-Beall2015}{}}%
\CSLLeftMargin{5. }%
\CSLRightInline{\textbf{Beall BFN}, \textbf{Twiss MR}, \textbf{Smith
DE}, \textbf{Oyserman BO}, \textbf{Rozmarynowycz MJ}, \textbf{Binding
CE}, \textbf{Bourbonniere RA}, \textbf{Bullerjahn GS}, \textbf{Palmer
ME}, \textbf{Reavie ED}, \textbf{Waters LMK}, \textbf{Woityra LWC},
\textbf{McKay RML}. 2015. Ice cover extent drives phytoplankton and
bacterial community structure in a large north-temperate lake:
Implications for a warming climate. Environmental Microbiology
\textbf{18}:1704--1719.
doi:\href{https://doi.org/10.1111/1462-2920.12819}{10.1111/1462-2920.12819}.}

\leavevmode\vadjust pre{\hypertarget{ref-Henson2016}{}}%
\CSLLeftMargin{6. }%
\CSLRightInline{\textbf{Henson MW}, \textbf{Pitre DM}, \textbf{Weckhorst
JL}, \textbf{Lanclos VC}, \textbf{Webber AT}, \textbf{Thrash JC}. 2016.
Artificial seawater media facilitate cultivating members of the
microbial majority from the gulf of mexico. {mSphere} \textbf{1}.
doi:\href{https://doi.org/10.1128/msphere.00028-16}{10.1128/msphere.00028-16}.}

\leavevmode\vadjust pre{\hypertarget{ref-Baxter2014}{}}%
\CSLLeftMargin{7. }%
\CSLRightInline{\textbf{Baxter NT}, \textbf{Wan JJ}, \textbf{Schubert
AM}, \textbf{Jenior ML}, \textbf{Myers P}, \textbf{Schloss PD}. 2014.
Intra- and interindividual variations mask interspecies variation in the
microbiota of sympatric peromyscus populations. Applied and
Environmental Microbiology \textbf{81}:396--404.
doi:\href{https://doi.org/10.1128/aem.02303-14}{10.1128/aem.02303-14}.}

\leavevmode\vadjust pre{\hypertarget{ref-LevyBooth2018}{}}%
\CSLLeftMargin{8. }%
\CSLRightInline{\textbf{Levy-Booth DJ}, \textbf{Giesbrecht IJW},
\textbf{Kellogg CTE}, \textbf{Heger TJ}, \textbf{D'Amore DV},
\textbf{Keeling PJ}, \textbf{Hallam SJ}, \textbf{Mohn WW}. 2018.
Seasonal and ecohydrological regulation of active microbial populations
involved in {DOC}, {CO}2, and {CH}4 fluxes in temperate rainforest soil.
The {ISME} Journal \textbf{13}:950--963.
doi:\href{https://doi.org/10.1038/s41396-018-0334-3}{10.1038/s41396-018-0334-3}.}

\leavevmode\vadjust pre{\hypertarget{ref-Edwards2015}{}}%
\CSLLeftMargin{9. }%
\CSLRightInline{\textbf{Edwards J}, \textbf{Johnson C},
\textbf{Santos-Medellín C}, \textbf{Lurie E}, \textbf{Podishetty NK},
\textbf{Bhatnagar S}, \textbf{Eisen JA}, \textbf{Sundaresan V}. 2015.
Structure, variation, and assembly of the root-associated microbiomes of
rice. Proceedings of the National Academy of Sciences
\textbf{112}:E911--E920.
doi:\href{https://doi.org/10.1073/pnas.1414592112}{10.1073/pnas.1414592112}.}

\leavevmode\vadjust pre{\hypertarget{ref-Ettinger2017}{}}%
\CSLLeftMargin{10. }%
\CSLRightInline{\textbf{Ettinger CL}, \textbf{Williams SL},
\textbf{Abbott JM}, \textbf{Stachowicz JJ}, \textbf{Eisen JA}. 2017.
Microbiome succession during ammonification in eelgrass bed sediments.
{PeerJ} \textbf{5}:e3674.
doi:\href{https://doi.org/10.7717/peerj.3674}{10.7717/peerj.3674}.}

\leavevmode\vadjust pre{\hypertarget{ref-Graw2018}{}}%
\CSLLeftMargin{11. }%
\CSLRightInline{\textbf{Graw MF}, \textbf{DAngelo G}, \textbf{Borchers
M}, \textbf{Thurber AR}, \textbf{Johnson JE}, \textbf{Zhang C},
\textbf{Liu H}, \textbf{Colwell FS}. 2018. Energy gradients structure
microbial communities across sediment horizons in deep marine sediments
of the south china sea. Frontiers in Microbiology \textbf{9}.
doi:\href{https://doi.org/10.3389/fmicb.2018.00729}{10.3389/fmicb.2018.00729}.}

\leavevmode\vadjust pre{\hypertarget{ref-Johnston2016}{}}%
\CSLLeftMargin{12. }%
\CSLRightInline{\textbf{Johnston ER}, \textbf{Rodriguez-R LM},
\textbf{Luo C}, \textbf{Yuan MM}, \textbf{Wu L}, \textbf{He Z},
\textbf{Schuur EAG}, \textbf{Luo Y}, \textbf{Tiedje JM}, \textbf{Zhou
J}, \textbf{Konstantinidis KT}. 2016. Metagenomics reveals pervasive
bacterial populations and reduced community diversity across the alaska
tundra ecosystem. Frontiers in Microbiology \textbf{7}.
doi:\href{https://doi.org/10.3389/fmicb.2016.00579}{10.3389/fmicb.2016.00579}.}

\leavevmode\vadjust pre{\hypertarget{ref-Hassell2018}{}}%
\CSLLeftMargin{13. }%
\CSLRightInline{\textbf{Hassell N}, \textbf{Tinker KA}, \textbf{Moore
T}, \textbf{Ottesen EA}. 2018. Temporal and spatial dynamics in
microbial community composition within a temperate stream network.
Environmental Microbiology \textbf{20}:3560--3572.
doi:\href{https://doi.org/10.1111/1462-2920.14311}{10.1111/1462-2920.14311}.}

\end{CSLReferences}

\bibliography{ref}
\setlength{\parindent}{0in}
\setlength{\leftskip}{0in}

\newpage

\textbf{Table 1. Summary of studies used in the analysis.} For all
studies, the number of sequences used from each study was rarefied to
the smallest sample size. A graphical represenation of the distribution
of sample sizes for each study and the samples that were removed from
each study are provided in Figure S1.

\small

\begin{longtable}[]{@{}
  >{\raggedright\arraybackslash}p{(\columnwidth - 10\tabcolsep) * \real{0.1367}}
  >{\raggedleft\arraybackslash}p{(\columnwidth - 10\tabcolsep) * \real{0.0664}}
  >{\raggedleft\arraybackslash}p{(\columnwidth - 10\tabcolsep) * \real{0.1914}}
  >{\raggedleft\arraybackslash}p{(\columnwidth - 10\tabcolsep) * \real{0.1953}}
  >{\raggedleft\arraybackslash}p{(\columnwidth - 10\tabcolsep) * \real{0.2031}}
  >{\raggedleft\arraybackslash}p{(\columnwidth - 10\tabcolsep) * \real{0.2070}}@{}}
\toprule()
\begin{minipage}[b]{\linewidth}\raggedright
\textbf{Study\nobreakspace{}(Ref)}
\end{minipage} & \begin{minipage}[b]{\linewidth}\raggedleft
\textbf{Samples}
\end{minipage} & \begin{minipage}[b]{\linewidth}\raggedleft
\makecell[c]{\textbf{Total}\\\textbf{sequences}}
\end{minipage} & \begin{minipage}[b]{\linewidth}\raggedleft
\makecell[c]{\textbf{Median}\\\textbf{sequences}}
\end{minipage} & \begin{minipage}[b]{\linewidth}\raggedleft
\makecell[c]{\textbf{Range of}\\\textbf{sequences}}
\end{minipage} & \begin{minipage}[b]{\linewidth}\raggedleft
\makecell[c]{\textbf{SRA study}\\\textbf{accession}}
\end{minipage} \\
\midrule()
\endhead
Bioethanol~(3) & 95 & 3,970,972 & 16,014 & 3,690\Hyphdash*356,027 &
SRP055545 \\
Human~(4) & 490 & 20,828,275 & 32,452 & 10,439\Hyphdash*422,904 &
SRP062005 \\
Lake~(5) & 52 & 3,145,486 & 69,205 & 15,135\Hyphdash*110,993 &
SRP050963 \\
Marine~(6) & 7 & 1,484,068 & 213,091 & 132,895\Hyphdash*256,758 &
SRP068101 \\
Mice~(1) & 348 & 2,785,641 & 6,426 & 1,804\Hyphdash*30,311 &
SRP192323 \\
Peromyscus~(7) & 111 & 1,545,288 & 12,393 & 4,454\Hyphdash*33,502 &
SRP044050 \\
Rainforest~(8) & 69 & 936,666 & 11,464 & 4,880\Hyphdash*37,403 &
ERP023747 \\
Rice~(9) & 490 & 22,623,937 & 43,399 & 2,777\Hyphdash*192,200 &
SRP044745 \\
Seagrass~(10) & 286 & 4,135,440 & 13,538 & 1,830\Hyphdash*45,076 &
SRP092441 \\
Sediment~(11) & 58 & 1,151,389 & 17,606 & 7,686\Hyphdash*67,763 &
SRP097192 \\
Soil~(12) & 18 & 932,563 & 50,487 & 46,622\Hyphdash*58,935 &
ERP012016 \\
Stream~(13) & 201 & 21,017,610 & 90,621 & 8,931\Hyphdash*394,419 &
SRP075852 \\
\bottomrule()
\end{longtable}

\normalsize

\newpage

\textbf{Figure 1.}

\newpage

\textbf{Figure S1.}

\end{document}
